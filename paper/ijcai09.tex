%%%% ijcai09.tex

\typeout{IJCAI-09 Instructions for Authors}

% These are the instructions for authors for IJCAI-09.
% They are the same as the ones for IJCAI-07 with superficical wording
%   changes only.

\documentclass{article}
% The file ijcai09.sty is the style file for IJCAI-09 (same as ijcai07.sty).
\usepackage{ijcai09}

% Use the postscript times font!
\usepackage{times}

% the following package is optional:
%\usepackage{latexsym} 

% Following comment is from ijcai97-submit.tex:
% The preparation of these files was supported by Schlumberger Palo Alto
% Research, AT\&T Bell Laboratories, and Morgan Kaufmann Publishers.
% Shirley Jowell, of Morgan Kaufmann Publishers, and Peter F.
% Patel-Schneider, of AT\&T Bell Laboratories collaborated on their
% preparation.

% These instructions can be modified and used in other conferences as long
% as credit to the authors and supporting agencies is retained, this notice
% is not changed, and further modification or reuse is not restricted.
% Neither Shirley Jowell nor Peter F. Patel-Schneider can be listed as
% contacts for providing assistance without their prior permission.

% To use for other conferences, change references to files and the
% conference appropriate and use other authors, contacts, publishers, and
% organizations.
% Also change the deadline and address for returning papers and the length and
% page charge instructions.
% Put where the files are available in the appropriate places.

\title{Predicting clinical features based on RNA-protein correlations in cancer patients}
\author{Hannah Boekweg, Corbin Day, Caleb Lindgren \\
Department of Biology\\
Brigham Young University}

\begin{document}

\maketitle

\begin{abstract}
  Your abstract should concisely answer the following three questions: 1) what problem are you addressing? 2) what approach are you taking to solve the problem? 3) what are your results?
\end{abstract}

\section{Introduction}
There are several molecular process that occur for proteins to be made. 
It begins with DNA (the instuctions for the cell), which gets transcribed to RNA, which is then able to be translated to proteins.
In cancer patients the correlation of RNA and it's corresponding protein are different than in healthy patients. We can use this to predict clinical features. 

\section{Methods}

\subsection{Baseline}

\subsection{Multi-layer perceptron}

\subsection{k-nearest neighbors}
We used sklearn's KNeighbors model. 10 cross fold validation was used and accuracy, precision, and recall scores were reported. 
We optimized the parameters using sklearn's RandomizedSearchCV.
The hyperparameter set for each paramter is as follows: n_neighbors=[stuff], weights=[stuff], algorithm=[stuff], leaf_size=[stuff], and p=[stuff]. 

\subsection{Naive Bayes}

\section{Results}

\section{Discussion}

%% The file named.bst is a bibliography style file for BibTeX 0.99c
\bibliographystyle{named}
\bibliography{ijcai09}

\end{document}

